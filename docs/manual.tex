\documentstyle{article} % Europeans might prefer [a4] as an option...
\begin{document}
\begin{titlepage}
\vspace{6.5in}
\begin{center}
{\huge MOVIE DATABASE\\
\Large v3.2}
\end{center}
\vspace{1in}

Welcome to the latest version of the database package. New features in
this release include:
\begin{itemize}
\item support for the seven new lists announced in December: literature,
    locations, movie links, release dates, tag lines, technical information
    and cast completion tracking.
\item Mark Harding's additions template code, making it easier to send 
    additions and corrections for the information in the database.
\item Easier matching of titles. You no longer need to worry about whether
    a titles needs a year, (TV) or (V) tag

\end{itemize}

Users of version 3.1 should see section~\ref{update} for details of the changes.

Comments, suggestions for new features, bug reports etc. via e--mail 
please.

\vspace{.5in}
\noindent Col Needham $<$col@imdb.com$>$
\end{titlepage}
\clearpage

\tableofcontents
\clearpage
\section{\label{intr}INTRODUCTION}

The movie database is an international organisation whose objective
is to provide useful and up to date movie information *freely* available 
on-line, across as 
many systems and platforms as possible. It currently covers over 75,000 movies
with over 1,000,000 filmography entries and is expanding continuously. The 
database includes filmographies for actors, directors, writers, composers, 
cinematographers, editors, production designers, costume designers and 
producers; plot summaries; character names; movie ratings; year of 
release; 
running times; movie trivia; quotes; goofs; soundtracks; personal trivia;
alternative names; certificates; color information; country of production;
genres; production companies; sound mix; reference literature; filming 
locations; sequel/remake information; release dates; advertising tag lines;
detailed technical data; and Academy Award information. Several thousand of 
the movies are covered completely from the major actors to the minor bit 
players. In their ASCII form, the source datafiles currently occupy about 32Mb
of disk-space. All the information comes from the users of the database so 
please continue to support it and help it to grow.

The movie database frequently asked questions list contains more 
information
on the whole movie database project and the different ways to access it. A 
copy is included in the package, alternatively send an e--mail message 
with the
subject:
\begin{quote}
{\tt HELP FAQ}
\end{quote}
to $<$mail-server@imdb.com$>$ and the movie mail--server will send a copy 
back. 
The database is available via an e--mail interface and on the World Wide 
Web and
full details are given in this FAQ.

\subsection{Requirements}

This version requires a Un*x like operating system, an ANSI C compiler and 
approximately 19Mb of disk-space for the full standard version. The package  
includes support for compressed database files using {\tt compress} or 
{\tt gzip}, 
bringing the requirements down to 12Mb of disk-space. There are options to
reduce the space further by discarding some of the databases that you might
not be interested in, so disk--space requirements can be as low as 5Mb.

You will need an ANSI compatible C compiler to compile the programs. If 
one 
wasn't supplied with your machine, I recommend the GNU C compiler gcc. 
You can pick up a source code distribution of gcc from ftp.gnu.ai.mit.edu
in pub/gnu which can be installed on a wide range of machines. NOTE FOR
USERS OF SUN MACHINES: the standard Sun C compiler is {\em not} ANSI 
compliant
so you will need a copy of gcc to install this package.

An Amiga version is also available both on AMINET (under biz/dbase) and 
from
the main anonymous FTP sites (see later).

An MS--DOS interface to the same data, but based on different software is 
also 
available from the FTP sites.

\subsection{\label{instl}Installation}

\begin{enumerate}
\item The database software is available from three FTP sites:
\begin{quote}
{\bf ftp.cathouse.org} in\\
/pub/cathouse/movies/database/tools/moviedb-3.2f.tar.gz\\
{\bf ftp.funet.fi} in\\
/pub/culture/tv+film/database/tools/moviedb-3.2f.tar.gz\\
{\bf ftp.fu-berlin.de} in\\
/pub/misc/movies/database/tools/moviedb-3.2f.tar.gz
\end{quote}
\item get the file using anonymous FTP and unpack by typing:
\begin{quote}
       {\tt gzip -cd moviedb-3.2f.tar | tar -xvf -}
\end{quote}
this will install the software in the directory moviedb-3.2, {\tt cd} into
the directory to continue.
\item since the package includes a program to automatically get the data 
files
from the FTP site (and to keep them up to date), you need to give the
package your e--mail address to use as the password for FTP. Edit the
file {\tt Makefile} and change the following line:
\begin{quote}
{\tt LFETCHOPT = -auto}
\end{quote}
to
\begin{quote}
{\tt LFETCHOPT = -auto -user $<$address$>$}
\end{quote}
where {\tt $<$address$>$} is your e--mail address.
\item follow the comments in the Makefile for further changes that might be
required for installation on your system. In particular, don't forget to
set {\tt CC} and/or {\tt CFLAGS} to enable ANSI compatible compilation.

Also, pay close attention to the variables which describe how to handle
the compressed list files ({\tt ZLISTCATCOM} etc.) and how to enable 
support
for compressed database files ({\tt ZDBSCATCOM} etc.)

The variables {\tt AUTOUNCOMPRESS} and {\tt KEEPLISTS} affect the amount 
of disk space used by the package if your resources are limited. Please 
refer to section~\ref{cstminst} for more information.
\item to compile all the programs, fetch the list files and create 
the databases run:
\begin{quote}
      {\tt make install}
\end{quote}

See section~\ref{mntng} if anything goes wrong with this process.

{\em NOTE:\/} the database generation process does require quite a bit 
memory 
and processing power to run. I've arranged to have ready made data
files available on the FTP sites for people who have problems generating
them locally. See section~\ref{rmdf} for more details.
\item the search programs will be installed in the moviedb-3.2/bin 
directory,
therefore you need to either add this directory to your path or run
all the programs from this directory.
\end{enumerate}

The package is now installed and you're ready to try it out, see the next
section for a quick introduction.

Section~\ref{mntFTP} has details on the one line command that you can use 
to 
ensure the database is always up to date with respect to the FTP site.

If you do not have direct access to anonymous FTP or you have some special
configuration requirements, see section~\ref{mntng} for more details on a 
customized set--up procedure and on fixing general problems.

\subsection{Quick Introduction}

If any of these commands fails to work, see section~\ref{mntng} for 
details of potential problems and how to resolve them.

The database supports many different kinds of queries and I refer you to 
the
rest of this manual for more information. This section is provided as a 
very simple introduction to searching:

To generate a filmography for a particular person including any personal
trivia, use:
\begin{quote}
   {\tt list -name $<$name$>$ -full}
\end{quote}
where $<$name$>$ is in "$<$surname$>$, $<$first name$>$" format. 

Examples:
\begin{quote}
   {\tt list -name "Hitchcock, Alfred" -full}\\
   {\tt list -name "Hamilton, Linda" -full}\\
   {\tt list -name "Grant, Cary" -full}
\end{quote}

To search for all the data on a particular movie (plot summary, year of
release, rating, cast, crew and trivia), use:
\begin{quote}
   {\tt title -t $<$title$>$ -full}
\end{quote}
bearing the following in mind:
\begin{itemize}
\item leading articles are placed at the end (e.g. "Terminator, The")
\item the year of release is used to distinguish between movies with the 
same
      or similar titles (e.g. "Cape Fear (1962)" and "Cape Fear (1991)")
\end{itemize}

Examples:
\begin{quote}
   {\tt title -t "Vertigo" -full}\\
   {\tt title -t "Aliens" -full}\\
   {\tt title -t "Jurassic Park" -full}
\end{quote}

The remainder of this document describes the search programs and 
maintenance
process in more detail. All the programs in the package output a brief 
usage 
description if run with no arguments.

\subsection{\label{dbs}Database Structure}

The database is made up of many files which are maintained by different 
people 
on the Internet. This section describes the files in more detail and gives
a contact address for the maintainer. The data can never hope to be 
complete
owing to the volume of movies being produced, therefore if you spot any
omissions, please let us know.

The most convenient way to send in new additions or corrections is via the
movie mail-sever's central collection service. For a complete submissions 
guide which covers everything send e-mail with the subject:
\begin{quote}
{\tt HELP ADD FULL}
\end{quote}
to $<$mail-server@imdb.com$>$ 

See the file {\tt CONTENTS} for a description of all the files in the 
package.

The titles in the database fall into one of five categories:
\begin{enumerate}
\item movies, unless otherwise indicated the title is a movie
\item TV--movies, indicated by a (TV) tag
\item made for video movies, indicated by a (V) tag
\item TV--series, indicated by enclosing the title in 's
\item mini--series, indicated by enclosing the title in 's and a (mini) tag
\end{enumerate}

\subsubsection{Cast Lists}

The character names information is published with the main list files,
but is still maintained separately by Steve Hammond. See 
table~\ref{tbl:cast}.

\begin{table}
\begin{center}
{\small
\begin{tabular}{|l|l|} \hline
List  & Maintained by  \\ \hline
Actors               & Col Needham $<$actors@imdb.com$>$ \\ \hline
Actresses            & Col Needham $<$actresses@imdb.com$>$ \\ \hline
Cast Completion      & Col Needham $<$col@imdb.com$>$ \\ \hline
\end{tabular}}
\caption{\label{tbl:cast} The cast databases}
\end{center}
\end{table}

Highlights from the actors and actresses lists are posted to 
rec.arts.movies on
a regular basis and contain filmographies for most popular performers in
those categories. The versions on the FTP sites contain the full lists with
all the actors and actresses known to the database.

Each filmography entry consists of up to five pieces of information: the 
name, the title, an attributes field (to store, for example, academy award 
nominations), the character name and a number to specify their position in
the credits.

Please note that guest appearances on TV-shows are not eligible for the 
cast
databases owing to the volume and accuracy of this type of information.

Note that the actors and actress lists only contain filmography 
information -
biographical data such as date and place of birth, real name etc. is stored
in the biographies list.

The cast completion list simply records which movies we have full cast lists
for in the database (i.e. everyone listed in the credits of the movie is in
the actors/actresses lists).

\subsubsection{Crew Lists}

The lists in table~\ref{tbl:crew} are of a similar format to the cast 
lists in 
that they contain filmographies for people in the respective fields.
\begin{table}
\begin{center}
{\small
\begin{tabular}{|l|l|} \hline
List  & Maintained by  \\ \hline
Cinematographers & Michel Hafner $<$cinematographers@imdb.com$>$ \\ \hline
Composers        & Michel Hafner $<$composers@imdb.com$>$ \\ \hline
Costume Designers & Harald Mayr $<$costumes@imdb.com$>$ \\ \hline
Directors        & Col Needham $<$directors@imdb.com$>$ \\ \hline
Misc. Filmography & Col Needham $<$miscellaneous@imdb.com$>$ \\ \hline
Editors          & Harald Mayr $<$editors@imdb.com$>$ \\ \hline
Producers        & Andre Bernhardt $<$producers@imdb.com$>$ \\ \hline
Production Designers & Harald Mayr $<$prod-designers@imdb.com$>$ \\ \hline
Writers          & Jon Reeves $<$writers@imdb.com$>$ \\ \hline
\end{tabular}}
\caption{\label{tbl:crew} The crew lists}
\end{center}
\end{table}

The miscellaneous filmography list is a ``catch all'' for people in 
careers not
covered by the other filmography lists. Anyone who worked on a movie can 
be 
listed here. The idea is to spin-off new filmography lists from it if new
list managers can be found and the volume of data relating to a particular
type of contribution justifies a new separate list.

\subsubsection{Titles Lists}

The lists in table~\ref{tbl:title} contain information that relates 
to individual titles.
\begin{table}
\begin{center}
{\small
\begin{tabular}{|l|l|} \hline
List  & Maintained by  \\ \hline
Alternative Titles  & Michel Hafner $<$aka-titles@imdb.com$>$ \\ \hline
Certificates & Mark Harding $<$certificates@imdb.com$>$ \\ \hline
Color Information & Mark Harding $<$color@imdb.com$>$ \\ \hline
Countries & Mark Harding $<$countries@imdb.com$>$ \\ \hline
Crazy Credits       & Mark Harding $<$crazy-credits@imdb.com$>$ \\ \hline
Genres              & Colin Tinto $<$genres@imdb.com$>$ \\ \hline
Goofs               & Murray Chapman $<$goofs@imdb.com$>$ \\ \hline
Literature          & Joachim Polzer $<$literature@imdb.com$>$ \\ \hline
Locations           & Rob Hartill $<$locations@imdb.com$>$ \\ \hline
Movie Links         & Col Needham $<$movie-links@imdb.com$>$ \\ \hline
Movies              & Michel Hafner $<$original-titles@imdb.com$>$ \\ \hline
Plot Summaries      & Colin Tinto $<$plots@imdb.com$>$ \\ \hline
Production Companies & Rob Hartill $<$prod-companies@imdb.com$>$ \\ \hline
Quotes              & Bob Glickstein $<$quotes@imdb.com$>$ \\ \hline
Ratings             & Col Needham $<$col@imdb.com$>$ \\ \hline
Release Dates       & Mark Harding $<$release-dates@imdb.com$>$  \\ \hline
Running Times       & Mark Harding $<$running-times@imdb.com$>$  \\ \hline
Sound Mix       & Mark Harding $<$sound-mix@imdb.com$>$  \\ \hline
Soundtracks         & Ron Higgins $<$soundtracks@imdb.com$>$ \\ \hline
Tag Lines            & Rob Hartill $<$tag-lines@imdb.com$>$ \\ \hline
Technical Info       & Joachim Polzer $<$technical@imdb.com$>$ \\ \hline
Trivia              & Murray Chapman $<$trivia@imdb.com$>$ \\ \hline
\end{tabular}}
\caption{\label{tbl:title} The lists relating to titles}
\end{center}
\end{table}

The alternative titles list contains foreign language, video titles, TV 
titles
etc. for the primary titles in the database.

The certificates list contains the certificates / ratings classifications
(MPAA, BBFC etc.)

The color information list records whether the movie was shot in color or
black \& white.

The countries list records the country of origin.

The crazy credits list contains trivia and jokes that appear in movie 
credits.

The genres list records records the genres which describe the movie 
(formally
part of the plot summaries list)

The goofs list contains a list of mistakes (or goofs) in movies.

The literature list records references to essays, critical analyses and books
relating to specific movies.

The locations list holds information on where movies were filmed.
 
The movie links list notes links between movies in the database such as 
sequels, remakes and also trivial connections such as homages and spoofs.

The movies list contains the year of release of most movies in the 
database,
along with any awards given to the movie as a whole.

The plot summaries list has brief outlines of some of the movies in the
database.

The production companies list records the production company / studio 
which 
produced the movie.

The quotes list contains memorable quotes from movies.

The ratings list is the rec.arts.movies Movie Ratings Report which is a
massive on--going electronic poll wherein readers of the group rate movies
on a scale from 1 (bad) to 10 (good).

The release dates list records the release dates of movies on a country by 
country basis.

The running times list contains the length in minutes of movies in the 
database.

The sound mix list records whether movies are recorded in mono, stereo etc.
or are silent.

The soundtracks list contains details of songs used on the soundtracks
of movies.

The tag lines list contains advertising tag lines used to publicize movies 
(e.g. for Alien "In space no-one can hear you scream")

The technical list records technical information such as aspect ratio, 
negative formats, print formats, cinematographic process etc.

The trivia list contains all kinds of trivia for movies from cameo 
appearances
to behind--the--scenes information.

\subsubsection{Biographies List}

The biographies list contains personal trivia on people in the database 
including dates of birth, dates of death, mini-biographies, agent's 
addresses
and books about them.

\subsubsection{Alternative Names List}

The alternative names list contains details of different names people have 
been billed under during their careers, for example Joanne Whalley / Joanne
Whalley-Kilmer.

\clearpage
\section{LIST PROGRAM}

The {\tt list} program searches for filmographies for people in the 
database.

\subsection{Database Options}

The {\tt -name} option searches all of the filmography lists for 
information.
This is rather inefficient if you know which list the person will be on. 
The
following options can be used in place of {\tt -name} to limit the search:
\begin{description}
\item[-cast]  search both cast databases
\item[-acr] search actors database
\item[-acs] search actresses database
\item[-dir]  search directors database
\item[-write]  search writers database
\item[-comp]  search composers database
\item[-cine] search cinematographers database
\item[-edit] search editors database
\item[-prodes] search production designers database
\item[-costdes] search costume designers database
\item[-prdcr] search producers database
\item[-misc] search miscellaneous database
\end{description}
 
Examples:
\begin{quote}
{\tt list -cast "Bogart, Humphrey"}\\
{\tt list -acr "Biehn, Michael"}\\
{\tt list -acs "Hepburn, Katharine"}\\
{\tt list -dir "Cameron, James"}
\end{quote}

\subsection{Output Options}

A basic {\tt list} query just generates a filmography for the person 
concerned. 
A number of output options are provided to link other information to the 
results:
\begin{description}
\item[-yr]    add the year of release to each title, and sort 
chronologically
\item[-yru]   add the year of release, but leave unsorted
\item[-mrr]   add movie ratings report information to each title
\item[-smrr]  add movie ratings report information, and sort by average 
rating
\item[-vmrr]  add movie ratings report information, and sort by number of 
votes
\item[-aka]   add any alternative titles / names and search aka names 
database
\item[-bio]   include information from biographies list and format output 
neatly
\item[-full]  combines {\tt -yr -chr -aka} and {\tt -bio}
\end{description}

Examples:
\begin{quote}
{\tt list -acr "Grant, Cary" -yr -chr}\\
{\tt list -dir "Hitchcock, Alfred" -smrr}
\end{quote}

\subsection{Control Options}

A couple of options are provided to control the search:
\begin{description}
\item[-m]       movies only: ignores TV--series and mini--series
\item[-i]       case sensitive search: normally searches are not case 
sensitive
\item[-genre $<$s$>$]  only include titles with a genre equal to $<$s$>$
\item[-time $<$s$>$]   only include titles with a time equal to $<$s$>$
\item[-prodco $<$s$>$] only include titles with a production company equal 
to $<$s$>$
\item[-cert $<$s$>$]   only include titles with a certificate equal to 
$<$s$>$
\item[-mix $<$s$>$]    only include titles with a sound mix equal to 
$<$s$>$
\item[-cntry $<$s$>$]  only include titles with a country equal to $<$s$>$
\item[-rel $<$s$>$]    only include titles with a release date equal to $<$s$>$
\item[-loc $<$s$>$]    only include titles with a location equal to $<$s$>$
\item[-tech $<$s$>$]   only include titles with technical info equal to $<$s$>$

\end{description}

The options may be combined to constrain the search further.

Example:
\begin{quotation}
\noindent{\tt list -dir "Hitchcock, Alfred" -prodco Selznick} \\
     list Hitchcock's movies made for Selznick studios

\noindent{\tt list -dir "Hitchcock, Alfred" -mix Silent} \\
     list Hitchcock's silent movies

\noindent{\tt list -dir "Lang, Fritz" -genre Film-Noir -full -color "Black 
and White"} \\
    list Fritz Lang's films noirs, shot in black and white

\end{quotation}

\subsection{\label{appr}Substring and Approximate Matching}

Ordinarily, the program requires an exact match in order to locate the 
information on a person. In some cases the exact spelling may not be known.
There are two classes of options which can help locate the specific person:
the first is substring matching, which treats the name you provided with 
the 
database option as a substring. The option to trigger this is {\tt -s}. 

The second class is approximate matching, which enables names with a 
limited 
number of character differences to still match. The options to trigger 
this 
are {\tt -1 -2 -3 -4 -5 -6 -7 -8} and {\tt -9} where the larger the 
number, 
the larger the number of differences allowed for a match. 

Examples:
\begin{quote}
{\tt list -cast "Barrymore" -s -full}\\
{\tt list -acr "rGant, Cary" -2 -full}
\end{quote}

\subsection{Multiple Name Queries}

Any number of database and name options may be combined in the same query
to generate a list of movies that all the people worked on together. 

Examples:
\begin{quote}
{\tt list -dir "Hitchcock, Alfred" -acr "Grant, Cary"}\\
{\tt list -acs "Hepburn, Katharine" -acr "Tracy, Spencer" -yr}
\end{quote}

\clearpage
\section{TITLE PROGRAM}

The title program searches the database for all the information relating
to a specific title. The basic option is {\tt -t $<$title$>$} which 
specifies 
movie to search for. If there is more than one movie in the database with
the same title, details on all matches are returned.

\subsection{Output Options}

A basic title query just returns the names of the people who worked on the 
movie. A number of options are provided to add further information to the
results:
\begin{description}
\item[-aka]    add any alternative titles
\item[-mrr]    add movie ratings report data
\item[-yr]     add year of release and any title attributes
\item[-pl]     add plot summary
\item[-trivia] add movie trivia, crazy credits, goofs, quotes, soundtrack 
info, certificates, color information, countries, genres, production 
companies, running times and sound mix.
\item[-full]   combines all of the above options
\end{description}

Examples:
\begin{quotation}
\noindent{\tt title -t "2001: A Space Odyssey" -full}\\

\noindent{\tt title -t "Cape Fear"}\\
returns details of both the 1962 and 1991 versions of this movie

\noindent{\tt title -t "Cape Fear (1991)"}\\
returns details of just the 1991 version

\end{quotation}


\subsection{Miscellaneous Options}

The following options are also available:
\begin{description}
\item[-i]       case sensitive search: normally searches are not case 
sensitive
\item[-s]       substring: match on substring rather than whole title
\end{description}

The {\tt -s} option is particularly useful for finding details of all the 
movies
with the same name. For example:
\begin{quote}
{\tt title -t "Thin Man" -full -s}
\end{quote}

\subsection{Multiple Title Searches}

Any number of {\tt -t} options may be combined in the same query to 
generate 
a list of people that worked on each and every one of the movies.

Example:
\begin{quote}
{\tt title -t "Ghostbusters" -t "Ghostbusters II"}
\end{quote}

\clearpage
\section{LISTALL PROGRAM}

This program can be used to generate filmographies for each person involved
with a particular movie.

Example:
\begin{quote}
{\tt listall -t "Abyss, The" -full}
\end{quote}

The following options are also available:
\begin{description}
\item[-yr]   add year of release to titles and sort chronologically
\item[-yru]  add year of release, but leave in alphabetical order
\item[-mrr]  add movie ratings report information
\item[-smrr] add movie ratings report and sort by average ratings
\item[-vmrr] add movie ratings report and sort by votes
\item[-chr]  add character names where available
\item[-bio]  add biography data
\item[-aka]  add alternative title data (and check if title is an aka 
title)
\item[-full] combines {\tt -aka -bio -chr -yr}
\item[-i] case sensitive search
\item[-m]    restrict search to movies only (ignore TV--series)
\end{description}

\clearpage
\section{LGUIDE PROGRAM}

This program works like {\tt title} except that it takes a list of titles 
from a 
file and searches for information on each one. I find {\tt lguide} really 
useful for generating a little movie guide to accompany the TV schedules 
for the local movie channels. To use the program simply type the titles 
into a file, one on each line and enter:
\begin{quote}
{\tt lguide -f $<$filename$>$}
\end{quote}

Other options:
\begin{description}
\item[-yr]   adds year of release information
\item[-mrr]  adds movie ratings report information
\item[-pl]   adds plot summary and formats results
\item[-trivia] add movie trivia, crazy credits, goofs, quotes, soundtrack 
info, certificates, color information, countries, genres, production 
companies and sound mix.
\item[-chr]  add character names
\item[-aka]  lists any alternative titles
\item[-full] combines all above options
\item[-i]    case sensitive search
\end{description}

A report of any titles which are not found in the database is written to 
the
standard error stream. 

\clearpage
\section{LTRACE PROGRAM}

This program displays all the information about each of the movies that 
someone
has worked on. The options are similar to those for {\tt list}:
\begin{description}
\item[-acr $<$name$>$]   search for actor
\item[-acs $<$name$>$]   search for actress
\item[-cast $<$name$>$]    search both cast lists
\item[-dir $<$name$>$]    search for director
\item[-write $<$name$>$]    search for writer
\item[-comp $<$name$>$]    search for composer
\item[-cine $<$name$>$]   search for cinematographer
\item[-edit $<$name$>$]   search for editor
\item[-prodes $<$name$>$]   search for production designer
\item[-costdes $<$name$>$]   search for costume designer
\item[-prdcr $<$name$>$]   search for producer
\item[-misc $<$name$>$]   search miscellaneous database
\item[-name $<$name$>$] search all filmography databases

\item[-yr]          adds year of release information
\item[-mrr]         adds movie ratings report information
\item[-pl]          adds plot summary and formats results
\item[-triv]        add movie trivia and formats results
\item[-chr]         add character names
\item[-aka]         lists any alternative titles
\item[-full]        combines this group of options
\item[-m]           movies only, ignore TV--series
\item[-i]           case sensitive search
\end{description}

For a good example try:
\begin{quote}
{\tt ltrace -dir "Hitchcock, Alfred" -full}
\end{quote}
finds the list of Hitchcock's movies then reports full details on each one.

\clearpage
\section{\label{lindex}LINDEX PROGRAM}

This program can be used to either to generate lists of names that appear 
on filmography lists or to generate lists of titles from the movies list.
It's uses include finding exact spellings of names or titles on the lists; 
finding names/titles containing a particular substring; and finding titles
released between a range of years or in a particular year.

\subsection{Database Options}

For title mode, simply use:
\begin{quote}
{\tt -title}
\end{quote}

For name mode, any combination of these options can be used:
\begin{description}
\item[-acr]    actors database
\item[-acs]    actresses database
\item[-dir]    directors database
\item[-write]  writers database
\item[-comp]   composers database
\item[-cine]   cinematographers database
\item[-edit]   editors database
\item[-prodes] production designers database
\item[-costdes] costume designers database
\item[-prdcr]  producers database
\item[-misc]  miscellaneous database
\end{description}

Note, if you do not specify any database options then the default is to 
search all the lists in name mode.

\subsection{Title Mode Options}

The following options are available only in title mode to limit the search
and to link the results with other information:
\begin{description}
\item[-yr]  adds the year of release and sorts in chronological order.
\item[-yru] adds the year of release, but leaves in alphabetical order.
\item[-yrfrom $<$yr$>$] limit search to titles released during or after 
$<$yr$>$
\item[-yrto $<$yr$>$]   limit search to titles released before $<$yr$>$
\item[-yreq $<$yr$>$]   limit search to titles released in $<$yr$>$
\item[-mrr]    add ratings report data
\item[-smrr]   add ratings report data and sort by average ratings
\item[-vmrr]   add ratings report data and sort by number of votes
\item[-match]  limit search to movies with entry in ratings database
\item[-veq $<$votes$>$]  limit search to movies with $<$votes$>$ votes
\item[-vmin $<$votes$>$]  limit search to movies with a minimum of 
$<$votes$>$ votes
\item[-vmax $<$votes$>$]  limit search to movies with a maximum of 
$<$votes$>$ votes
\item[-m]      limit search to movies only, i.e. no TV--series
\item[-genre $<$s$>$]  limit search to titles with a genre equal to $<$s$>$
\item[-time $<$s$>$]  limit search to titles with a time equal to $<$s$>$
\item[-prodco $<$s$>$]  limit search to titles with a production company 
equal to $<$s$>$
\item[-cert $<$s$>$]  limit search to titles with a certificate equal to 
$<$s$>$
\item[-mix $<$s$>$]  limit search to titles with a sound mix equal to 
$<$s$>$
\item[-cntry $<$s$>$]  limit search to titles with a country equal to 
$<$s$>$
\item[-rel $<$s$>$]  limit search to titles with a release date equal to $<$s$>$
\item[-loc $<$s$>$]  limit search to titles with a location equal to $<$s$>$
\item[-tech $<$s$>$]  limit search to titles with technical info equal to $<$s$>$
\end{description}

These options maybe combined to limit the search to titles matching *all*
the constraints.

\subsection{Miscellaneous Options}

\begin{description}
\item[-s $<$substring$>$]  restrict search to names/titles containing 
$<$substring$>$
\item[-i]  become case sensitive with {\tt -s}
\end{description}

\subsection{Examples}

\begin{quotation}
\noindent{\tt lindex -acs -acr -name}\\
   generate list of all the names on the cast lists
   
\noindent{\tt lindex -acr -s "Michael" -name}\\
   generate list of actors containing the substring ``Michael''
  
\noindent{\tt lindex -title -s "Alien"}\\
   generate list of all movie/tv titles containing ``Alien'' 
  
\noindent{\tt lindex -title -yr -yreq 1945}\\
   list all titles in database released in 1945
  
\noindent{\tt lindex -title -yr -yrfrom 1990 -yrto 1994 -smrr}\\
   sort all movies released in the 1990s by average rating

\noindent{\tt index -title -genre Film-Noir -prodco RKO}\\
   list films noirs made for RKO

\noindent{\tt lindex -title -prodco MGM -color Color -yrto 1940}\\
   list all color MGM movies released up to 1940
\end{quotation}

\clearpage
\section{\label{alist}ALIST PROGRAM}

This program searches the attribute fields of all the databases. Examples
of the kind of queries you can construct include:
\begin{itemize}
\item search for details of all best actor academy award winners
\item search for movies nominated for best picture
\item search for all movies narrated by a particular person
\item search for all plays by a particular writer that have been filmed
\end{itemize}

\subsection{Database Options}

For title mode, simply use:
\begin{quote}
{\tt -title}
\end{quote}

For name mode: 
\begin{description}
\item[-acr]    actors database
\item[-acs]    actresses database
\item[-cast]   search both cast databases
\item[-dir]    directors database
\item[-write]  writers database
\item[-comp]   composers database
\item[-cine]   cinematographers database
\item[-edit]   editors database
\item[-prodes] production designers database
\item[-costdes] costume designers database
\item[-prdcr]  producers database
\item[-misc]  miscellaneous database
\end{description}

Only one database option is allowed in each run.

\subsection{Search Options}

These options specify the information to be searched for:
\begin{description}
\item[-aa] academy award winners
\item[-aan] academy award nominees
\item[-gg] golden globe winners
\item[-ggn] golden globe nominees
\item[-attr $<$pattern$>$] values matching $<$pattern$>$
\end{description}

The academy award options can be combined together to give a list of all 
winners and nominees. The same is true for the golden globe options.

These options can be used to limit the search in various ways:
\begin{description}
\item[-m]                 restrict search to movies only
\item[-yrfrom $<$yr$>$]   restrict search to titles released during or 
after $<$yr$>$
\item[-yrto $<$yr$>$]     restrict search to titles released before 
$<$yr$>$
\item[-yreq $<$yr$>$]     restrict search to titles released in $<$yr$>$
\end{description}

\subsection{Miscellaneous Options}

\begin{description}
\item[-s]   substring, allow partial matches with {\tt -attr}
\item[-i]   become case sensitive in {\tt -attr} searches
\item[-yr]  add year of release to titles and group results by year
\end{description}

\subsection{Examples}

\begin{quotation}
\noindent{\tt alist -dir -aa -aan -yr}\\
 search directors database for academy award winners/nominees and sort
 by year
 
\noindent{\tt alist -title -aa -aan -yr}\\
 search for best picture academy award winners/nominees and sort by year
 
\noindent{\tt alist -title -attr "(D)" -s -yr}\\
 search for all documentaries in the database
 
\noindent{\tt alist -aa -dir -yr -yrfrom 1940 -yrto 1949}\\
 lists all best director oscar winners of the 1940s by year
\end{quotation}

\clearpage
\section{\label{mkballot}MKBALLOT PROGRAM}

This program generates a movie rating ballot for all the movies a 
particular person has been involved with. The options conform to 
the standards used in the {\tt list} program:

\begin{description}
\item[-acr] actor
\item[-acs] actress
\item[-dir] director
\item[-write] writer
\item[-comp] composer
\item[-cine] cinematographer
\item[-edit] editor
\item[-prodes] production designer
\item[-costdes] costume designer
\item[-prdcr] producer
\item[-misc] miscellaneous
\end{description}

The results should be redirected to a file where you can add in votes for
the movies you have seen and then mail the file to the vote tabulator. For
further details on taking part in the movie ratings poll see the weekly
postings in rec.arts.movies.

Example:
\begin{quote}
{\tt mkballot -acr "Bogart, Humphrey"}\\
generate a ballot for all Bogie's movies
\end{quote}

\clearpage
\section{LFORMAT PROGRAM}

This program provides a number of features for formatting whole databases,
combining the information with the movies list or the movie ratings 
report. 

\subsection{Database options}

These options specify which database you would like to format:
\begin{description}
\item[-acr] actors database
\item[-acs] actresses database
\item[-dir] directors database
\item[-write] writers database
\item[-comp] composers database
\item[-cine] cinematographers database
\item[-edit] editors database
\item[-prodes] production designers database
\item[-costdes] costume designers database
\item[-prdcr] producers database
\item[-misc] miscellaneous database
\end{description}

\subsection{Format Options}

\begin{description}
\item[-yr] Adds the year of release information and sorts each person's 
entries into chronological order.
\item[-yru] Adds the year of release information, but leaves entries in
alphabetical order.
\item[-mrr] Adds the information from the movie ratings report to each 
title.
\item[-smrr] Adds the information from the movie ratings report to each 
title 
and sorts each person's entries by average rating.
\item[-vmrr] Adds the information from the movie ratings report to each 
title 
and sorts each person's entries by number of votes.
\end{description}

If no format option is given, the program formats the results in the style 
of the published lists.

\subsection{Output Options}

\begin{description}
\item[-r]     raw mode, don't format results
\item[-m]     movies only, ignores TV--series
\end{description}

\subsection{Examples}

\begin{quotation}
\noindent{\tt lformat -acr} \\
     regenerates published form of actors list
     
\noindent{\tt lformat -dir -yr}\\
     formats directors database \& adds year information
     
\noindent{\tt lformat -acs -mrr}\\
     formats entire actress list and adds ratings report information
\end{quotation}

\clearpage
\section{LSEEN PROGRAM}

This program is for use in conjunction with my movie ratings report voting 
support scripts. This program counts/sorts/formats lists, extracting films 
you 
have seen based on data from your votelog file.

If you don't use the voting support scripts, they are available via 
anonymous
FTP from ftp.cathouse in pub/cathouse/movies/database/tools/voting\_scripts.gz.

If you'd like a copy of your votelog file from the movie ratings database,
contact me via e--mail and I'll send it to you.

\subsection{Creating the Votes Database}

Place a copy of the votelog file in:
\begin{quote}
{\tt moviedb-3.2/lists/votelog}
\end{quote}
then from the moviedb-3.2 directory type:
\begin{quote}
{\tt make votes}
\end{quote}

In order to get maximum benefit from {\tt lseen}, titles in votelog need 
to be
compatible with those in the movies list (moviedb-3.2/lists/movies.list). A
list of the incompatible titles will be displayed each time you do a 
{\tt make votes}. Titles are considered compatible if they match apart 
from 
their case (titles in votelog should be in lowercase) and (TV)/(V) tags
are not needed in votelog.

\subsection{Database Options}

These options specify which database you would like to use:
\begin{description}
\item[-acr] actors database
\item[-acs] actresses database
\item[-dir] directors database
\item[-write] writers database
\item[-comp] composers database
\item[-cine] cinematographers database
\item[-edit] editors database
\item[-prodes] production designers database
\item[-costdes] costume designers database
\item[-prdcr] producers database
\item[-misc] miscellaneous database
\end{description}

\subsection{Output Options}

\begin{description}
\item[-us] unsorted, don't sort data by total movies you have seen
\item[-yr] add the year of release to each title, and sort chronologically
\item[-yru] add the year of release, but leave unsorted
\item[-mrr] add movie ratings report info
\item[-smrr] add movie ratings report info, sort by average ratings
\item[-vmrr] add movie ratings report info, sort by number of votes
\end{description}

\subsection{Control Options}

\begin{description}
\item[-min $<$value$>$] limit output to people who you've seen at least 
$<$value$>$ of their movies
\end{description}

\subsection{Examples}

\begin{quotation}
\noindent{\tt lseen -dir } \\
count directors

\noindent{\tt lseen -acr -mrr -yr} \\
count actors and display ratings report and year info for each movie
\end{quotation}

\clearpage
\section{\label{mntng}MAINTAINING THE DATABASE}

The database can be maintained entirely via the {\tt Makefile} in the top 
level
moviedb-3.2 directory. All the {\tt make} commands in this section should 
be run 
from this directory. See the comments at the top of the file for a full 
list 
of all the options supported, including installing the man pages. Many 
thanks 
to Phil Queinnec $<$queinnec@dgac.fr$>$ for the Makefiles.

The file {\tt docs/CONTENTS} contains a list of all the files in the 
package and a
brief description of their purpose for those interested.

\subsection{\label{rmdf}Ready Made Database Files}

The program which generates the internal database files from the ascii text
list files requires quite a bit memory and processing power to run. I've 
tried
to keep the requirements to a minimum, but there may be problems on some
machines. I've therefore arranged for ready made database files to be 
available via FTP in addition to the source list files. You can now choose
whether to install from the list files and run the generation process 
locally,
or to just install the ready made databases instead. The files have been
wrapped in a {\tt tar} archive and are available via anonymous FTP:
\begin{quote}
{\bf ftp.cathouse.org} in\\
/pub/cathouse/movies/database/tools/ready-made-data.tar.gz\\
{\bf ftp.funet.fi} in\\
/pub/culture/tv+film/database/tools/ready-made-data.tar.gz
\end{quote}

The ready made database files will only be updated at most monthly so if 
you 
want the latest data available, I recommend continuing to use the list 
files 
(the default). If you do want to use the ready made files, follow these 
alternative installation instructions:
\begin{enumerate}
\item follow steps (1) to (4) as described in section~\ref{instl}
\item from the moviedb-3.2 directory run:
\begin{quote}
{\tt make all}
\end{quote}
\item get the ready made database file from the nearest FTP site as listed 
above, place it in your moviedb-3.2 directory and run:
\begin{quote}
{\tt gzip -cd ready-made-data.tar | tar -xvf -}
\end{quote}
\item The files should now be installed so continue to follow the standard
      installation instructions from step (6) in section~\ref{instl}.
\end{enumerate}

\subsection{\label{mntFTP}Updating via FTP}

If you have open access to the internet via FTP, the database can be 
maintained
completely automatically. Simply type:
\begin{quote}
{\tt make update}
\end{quote}
and the system will check the FTP site to see if any of the files have been
modified since you last did an update, get them automatically and 
regenerate
the appropriate database files. Doing this every couple of weeks or 
whenever
you see an announcement in r.a.movies about a list being updated will 
ensure
that you always have the latest files.

If you don't have direct access to the FTP site or aren't comfortable with
the system initiating FTP transfers behind your back: manually place copies
of the updated list files in the database directory and type:
\begin{quote}
{\tt make update-local}
\end{quote}

{\em NOTE:\/} Writing an automatic FTP program that runs on most machines 
has 
proved very difficult so on some systems you'll have to manually get the 
files.  

\subsection{Updating from rec.arts.movies}

Most of the posted lists are split over several smaller articles. You will
need to join all the articles together and remove the news headers from
part two and any subsequent parts. Do not delete any of the header text 
from 
part one since the system needs it to detect the start of data. Save the 
results in the filename indicated in the table~\ref{tbl:fnames}.
\begin{table}
\begin{center}
\begin{tabular}{|l|l|} \hline
List & List File \\ \hline
Actors & actors.list \\ 
Actresses & actresses.list\\
Biographies & biographies.list\\
Cinematographers & cinematographers.list\\
Composers & composers.list \\
Crazy Credits & crazy-credits.list\\ 
Directors & directors.list\\
Goofs & goofs.list\\
Movie Ratings & ratings.list\\
Quotes & quotes.list\\
Soundtracks & soundtracks.list\\
Trivia & trivia.list\\
Writers & writers.list\\ \hline
\end{tabular}
\caption{\label{tbl:fnames} Filenames for the Lists}
\end{center}
\end{table}

Then type:
\begin{quote}
{\tt make update-local}
\end{quote}

Lists not mentioned in table~\ref{tbl:fnames} are only available from the 
anonymous
FTP sites. Note: the posted versions of the actors, actresses and 
biographies
lists only contain highlights of the data available -- the full versions 
are
only on the FTP sites.

\subsection{\label{cstminst}Customized Installation}

The standard installation assumes that you have direct and open access to
the FTP site via the {\tt ftp} command and that you have 47Mb of free 
disk--space to store the database. In reality, I don't expect these 
assumptions to be true in the majority of cases. This section describes 
how to modify the configuration and how to reduce the disk--space 
requirements.

\subsubsection{Reducing Space Requirements}

By default the system uncompresses all the list files before processing 
them.
To leave them in their compressed form, set the top--level Makefile 
variable
{\tt AUTOUNCOMPRESS} as described in the comments. This represents a 
saving 
of 16Mb and only affects the performance at database generation time -- 
it doesn't affect the searching of the database later.

Once a database file has been created, you do not need to keep the 
corresponding source list file. The surplus list files can be deleted by:
\begin{quote}
{\tt make cleanlists}
\end{quote}

This step reduces the space requirements by a further 11Mb. If you set the
variable {\tt KEEPLIST} in the Makefile as directed in the comments, 
the system will automatically delete the lists each time it has processed 
them. 

The next thing to consider is compressing some or all of the databases. The
programs can all search compressed databases transparently although 
performance does suffer as a result. Many thanks to Timo Lamminjoki 
$<$lamminjo\-@fltxa.\-helsinki.\-fi$>$ for the code to do this. To 
compress 
the main database files use:
\begin{quote}
{\tt make compress}
\end{quote}

This saves an additional 7Mb. If space is an absolute premium you can save 
a
further 500K by using:
\begin{quote}
{\tt make maxcompress}
\end{quote}
to compress the index files too (at the cost of a further performance hit).
These figures assume you are using {\tt gzip} to compress the database -- 
it is
faster and compresses better than {\tt compress}. 

All these steps bring the disk--space requirements for the database files 
down 
to just 12Mb. The following steps now save space by reducing the amount of
information in the database.

Another saving of about 1Mb (uncompressed) can be made if you are not 
interested in storing TV--series information: edit the Makefile changing 
the
line:
\begin{quote}
{\tt MKDBOPT = }
\end{quote}
to
\begin{quote}
{\tt MKDBOPT = -m}
\end{quote}

A saving of 2Mb (uncompressed) can be made if you aren't interested in
the character names information: edit the Makefile and add the option
{\tt -nochar} to the MKDBOPT variable:

\begin{quote}
{\tt MKDBOPT = -nochar}
\end{quote}

Finally, to remove any databases you aren't interested in, {\tt cd} into
moviedb-3.2/dbs and delete the associated files, for example to eliminate
the editors:

\begin{quote}
{\tt rm editors.*}\\
\end{quote}
then from the top-level directory do:
\begin{quote}
{\tt make touch-dbs}
\end{quote}
to create empty versions of the deleted files. Please make sure you delete
all the files associated with a particular list by using the wildcard {\tt 
*} 
for the filename extension as above.

\subsubsection{FTP set--up}

If you only have access to FTP via indirect commands such as {\tt 
gate-ftp} or
wish to modify the ftp parameters in some way, edit the file {\tt lfetch}. 
In
the area after the initial comment block there are some variable 
assignments
which you can edit to suit your system.

\subsection{Troubleshooting}

This section will be expanded according to the feedback I receive on how
easy the package is to install and what the common problems are. I 
recommend
reading section~\ref{cstminst} first! If you are still having problems 
with the package after reading this section, contact me via e-mail: 
$<$col@imdb.com$>$.

\begin{description}
\item[Q1:] The automatic list collection doesn't work on my machine even 
after
     adjusting the configuration variables?
\item[A1:] Get the files manually and run:
\begin{quote}
{\tt make update-local}
\end{quote}
\item[Q2:] The system fails to create the database files from the 
compressed list
     files?
\item[A2:] Get the files automatically (if you can) by running:
\begin{quote}
{\tt make do-lfetch}
\end{quote}
or get the files manually and then run:
\begin{quote}
{\tt gzip -d *.gz}\\
{\tt make update-local}
\end{quote}
\item[Q3:] The FTP transfer failed part--way through an update?
\item[A3:] Enter the following (in the moviedb-3.2/lists directory):
\begin{quote}
{\tt mv filesizes.old filesizes}
\end{quote}
to reset the local state and start the update again.
\item[Q4:] How do I make the auto--fetch program get a file from the FTP 
site even
     when it thinks the local and remote files are the same?
\item[A4:] From the lists directory run: 
\begin{quote}
{\tt ../etc/lfetch -force -user $<$e-mail address$>$}
\end{quote}
together with the appropriate option(s) as described in the comments in
the {\tt lfetch} file.
\item[Q5:] I deleted all the list and database files but when I do a {\tt 
make 
update} the program says that all the local files are up to date?
\item[A5:] The system keeps track of which lists you have updated by 
recording their
sizes in a file when they are downloaded. To reset this file use:
\begin{quote}
{\tt make reset-ftp}
\end{quote}
\item[Q6:] Whenever I do a search the programs don't return any results or 
I get
several error messages of the form:
\begin{quote}
{\tt ../moviedb-3.2/dbs/actors.dbs.gz: No such file or directory}
\end{quote}
\item[A6:] The automatic fetching program probably failed to collect the 
list
files and therefore the databases could not be created. See ({\bf A1}) to
fix this.
\item[Q7:] The database generation program {\tt mkdb} fails to create the 
database files.
\item[A7:] Use the ready made database files instead (see 
section~\ref{rmdf} 
for details)
\end{description}

\subsection{Logfiles}

If the database is installed on a multi-user machine and you'd like to keep
track of the usage, the programs can append various details to a system
log file. To enable this feature see the "LOGFILENAME" variable in the top
level makefile.
\clearpage
\section{\label{update}CHANGES FOR USERS OF VERSION 3.1}

This section describes changes for users of the previous version.

\subsection{Upgrading to version 3.2}

There are two upgrade options:

\begin{enumerate}
\item If you have all the .list files

\renewcommand{\theenumii}{\roman{enumii}}
\begin{enumerate}
\item follow steps (1), (2), (3) and (4) in section~\ref{instl}

\item copy all the list files (.list extenstions) and the {\tt filesizes} file
from {\tt moviedb-3.1/lists} to the directory {\tt moviedb-3.2/lists}. 

\item to compile all the programs, fetch the new/updated list files and create 
       the databases run:
\begin{quote}
{\tt make install}
\end{quote}
       from the top level {\tt moviedb-3.1} directory.
\item the search programs will be installed in the {\tt moviedb-3.2/bin} directory,
       therefore you need to either add this directory to your path or run
       all the programs from this directory.
       
\item if all goes well you can now delete the old {\tt moviedb-3.1} directory

\end{enumerate}

If the automatic FTP fetching script fails to work in step (iii) you will need
to manually get the new and updated files from one of the FTP sites, put
them in the lists directory and do 'make install-local' from the top-level
{\tt moviedb-3.2} directory.

\item If you just have the database files
\renewcommand{\theenumii}{\roman{enumii}}
\begin{enumerate}
\item follow steps (1), (2), (3) and (4) in section~\ref{instl}

\item copy all the database files (everything in {\tt moviedb-3.1/dbs} 
{\em except} the
       file {\tt moviedb-3.1/dbs/Makefile} - you {\em must} delete or rename this file 
       before issuing the copy command) to the directory {\tt moviedb-3.2/dbs}.
       
       
\item copy the file {\tt moviedb-3.1/lists/filesizes} to {\tt moviedb-3.2/lists/filesizes}

\item to compile all the programs, fetch the new/updated list files and create 
       the databases run:
\begin{quote}
{\tt make install}
\end{quote}
       from the top level {\tt moviedb-3.2} directory.
       
\item the search programs will be installed in the {\tt moviedb-3.2/bin} directory,
       therefore you need to either add this directory to your path or run
       all the programs from this directory.
       
\item  if all goes well you can now delete the old {\tt moviedb-3.1} directory

\end{enumerate}

If the automatic FTP fetching script fails to work in step (iv) you will need
to manually get the new and updated files from one of the FTP sites, put
them in the lists directory and do 'make install-local' from the top-level
{\tt moviedb-3.2} directory.

See section~\ref{mntng} for details on how to maintain the databases in this version.
\end{enumerate}                                

\subsection{New List and Options}
The data from the new lists will be added to any 'title -full' query. For
good examples of the new data in operation try:
\begin{quote}
{\tt title -t "2001: A Space Odyssey" -full}
\end{quote}
Changes to the options / functionality include:
\begin{itemize}
\item new title options on 'lindex' to generate lists of titles with specific
   locations, release dates and technical data. See section 7.2 for a list.

\item ability to constrain filmography searches by locations, release dates and
   technical data. See section 2.3 for details.

\item support for logging the database usage, see section 12.6

\item you no longer need to worry about whether a title needs a year, (TV) or
   (V) tags for the 'title' program. If more than one match exists the 
   program returns a list of matching titles and the details on each one.   
   
For example:
\begin{quotation}
\noindent{\tt title -t "Man Who Knew Too Much, The" -full}\\
returns data on both the 1956 and 1934 versions

\noindent{\tt title -t "Man Who Knew Too Much, The (1956)" -full}\\
returns data on just the 1956 version 
\end{quotation}
\end{itemize}
\clearpage   
\section{ADDING TO THE DATABASE}
The database depends on its users to keep it accurate and up to date. A copy
of the complete additions guide is included with the software in the file:
\begin{quote}
{\tt moviedb-3.2/docs/ADDS-GUIDE}
\end{quote}
Mark Harding's template generation software is also included in the
distribution. The programs are 'template' for title based templates and 
'templist' for name based templates. The usage is:
\begin{quotation}
\noindent{\tt UPDATE TITLE }\\
\noindent{\tt UPDATE NAME}
\end{quotation}
for title and name based additions respectively.

For example:
\begin{quotation}
\noindent{\tt title -t "Vertigo" -full | template > adds.vertigo}\\
Edit the file {\tt adds.vertigo} with your favorite editor.
 
\noindent{\tt mail -s "UPDATE TITLE" mail-server@imdb.com < adds.vertigo}
\end{quotation}
\clearpage  
\appendix
\section{KEY TO THE SYMBOLS USED IN THE DATABASE}

\subsection{Key to the titles}
The titles in the database fall into one of five categories:
\begin{enumerate}
\item movies, unless otherwise indicated the title is a movie
\item TV--movies, indicated by a (TV) tag
\item made for video movies, indicated by a (V) tag
\item TV--series, indicated by enclosing the title in 's
\item mini--series, indicated by enclosing the title in 's and a (mini) tag
\end{enumerate}

\newenvironment{Ventry}[1]%
{\begin{list}{}{\renewcommand{\makelabel}[1]{\textsf{##1}\hfil}%
\settowidth{\labelwidth}{\textsf{#1}}%
\leftmargin=\labelwidth \advance \leftmargin by \labelsep}}%
{\end{list}}
\subsection{Key to the symbols appearing in filmographies}

The filmography databases enable details of awards earned for or notes on a 
particular performance/contribution to be recorded. Each database has a 
different set of symbols, although some overlap where appropriate. 

\subsubsection{Universal Symbols}
\begin{Ventry}{(Video Premiere)}
\item[(AA)]   Academy Award winner in relevant category
\item[(AAN)]          Academy Award nomination in relevant category
\item[(GG)]           Golden Globe winner in relevant category
\item[(GGN)]            Golden Globe nomination winner in relevant category
\item[(BFA)]           British Academy Award ( = British Film Award )
\item[(Cesar)]        French Academy Award 
\item[(David)]         Italian Academy Award 
\item[(Felix)]         European Academy Award ( = European Film Award )
\item[(Guldbagge)]      Swedish Academy Award
\end{Ventry}

\subsubsection{Acting Key}
\begin{Ventry}{(Video Premiere)}
\item[(S:AA)]  academy award winner (best supporting actor/actress)
\item[(S:AAN)]         academy award nominee (best supporting actor/actress)
\item[(C:GG)]          golden globe winner (best comedy/musical actor/actress)
\item[(C:GGN)]         golden globe nominee (best comedy/musical actor/actress)
\item[(S:GG)]          golden globe winner (best supporting actor/actress)
\item[(S:GGN)]         golden globe nominee (best supporting actor/actress)
\item[(N:GG)]          golden globe winner (best newcomer)
\end{Ventry}

\subsubsection{Writers Key}

\begin{Ventry}{(Video Premiere)}
\item[(A:AA)]      Academy Award for Best Adaptation
\item[(A:AAN)]     Academy Award Nomination for Best Adaptation
\item[(DS:AA)]     Academy Award for Best Screenplay Written Directly
 for the Screen
\item[(DS:AAN)]    Academy Award Nomination for Best Screenplay Written  
Directly for the Screen
\item[(MPS:AA)]    Academy Award for Best Motion Picture Story
\item[(MPS:AAN)]   Academy Award Nomination for Best Motion Picture Story
\item[(S:AA)]      Academy Award for Best Screenplay
\item[(S:AAN)]     Academy Award Nomination for Best Screenplay
\item[(SA:AA)]     Academy Award for Best Screenplay -- Adapted
\item[(SA:AAN)]    Academy Award Nomination for Best Screenplay -- Adapted
\item[(SAM:AA)]    Academy Award for Best Screenplay -- based on Another Medium
\item[(SAM:AAN)]   Academy Award Nomination for Best Screenplay 
-- based on Another Medium
\item[(SAO:AA)]    Academy Award for Best Screenplay Adapted from Other Material
\item[(SAO:AAN)]   Academy Award Nomination for Best Screenplay Adapted 
from Other Material
\item[(SO:AA)]     Academy Award for Best Screenplay -- Original
\item[(SO:AAN)]    Academy Award Nomination for Best Screenplay -- Original
\item[(SS:AA)]     Academy Award for Best Story and Screenplay
\item[(SS:AAN)]    Academy Award Nomination for Best Story and Screenplay
\item[(SSF:AA)]    Academy Award for Best Story and Screenplay -- based 
on Factual material or material not previously published or produced
\item[(SSF:AAN)]   Academy Award Nomination for Best Story and Screenplay -- 
 based on Factual material or material not prev. published or produced
\item[(SSS:AA)]    Academy Award for Best Story and Screenplay -- written
 directly for the Screen
\item[(SSS:AAN)]   Academy Award Nomination for Best Story and Screenplay 
-- written directly for the Screen
\item[(StO:AA)]    Academy Award for Best Story -- Original
\item[(StO:AAN)]   Academy Award Nomination for Best Story -- Original
\item[(T:AA)]      Academy Award for Title Writing
\item[(T:AAN)]     Academy Award Nomination for Title Writing
\item[(WA:AA)]     Academy Award for Writing Achievement
\item[(WA:AAN)]    Academy Award Nomination for Writing Acheivement
\item[(BA)]        British Academy of Film and Television Arts for 
Best Screenplay
\item[(G)]         Genie for Best Screenplay
\item[(GG)]        Golden Globe for Best Screenplay
\end{Ventry}

\begin{Ventry}{(Video Premiere)}
\item[(song)]   The writer is the author (composer?) of a song on
 which the screenplay was based.
\item[(story)]         The writer contributed on the story for the
 screenplay only or the writer is the author of a
 story on which the movie was based.
\item[(play)]          The writer is the author of the play on which the
 screenplay was based.
\item[(novel)]         The writer is the author of the novel on which the
 screenplay was based.
\item[(novel XXX)]     The writer is the author of the novel on which the
 screenplay was based, but the novel is not the same
 title as the movie.
\item[(book)]          The writer is the author of the book on which the
 screenplay was based.
\item[(article)]       The writer is the author of the article on which the
 screenplay was based.
\item[(operetta)]      The writer is the author of the operetta on which the
 screenplay was based.
\item[(also XXX)]      The writer was involved in some other aspect than
 just writing the screenplay.  They may have authored the novel
 the screenplay was based on, or developed the story, etc.
\end{Ventry}
\subsubsection{Composers Key}
\begin{Ventry}{(Video Premiere)}
\item[(A:AA)]  Academy Award for best score adapted from another musical 
                source
\item[(A:AAN)]        Academy Award Nomination for best score adapted from
                 another musical source
\item[(AAA)]          Australian Academy Award
\item[(D:AA)]         Academy Award for best score in a drama or a comedy
\item[(D:AAN)]        Academy Award Nomination for best score in a drama 
or a comedy
\item[(M:AA)]         Academy Award for best score in a musical
\item[(M:AAN)]        Academy Award Nomination for best score in a musical
\item[(O:AA)]         Academy Award for best original score
\item[(O:AAN)]        Academy Award Nomination for best original score
\item[(BS:AA)]        Academy Award for best song
\item[(BS:AAN)]       Academy Award Nomination for best song
\item[(songs)]        The composer composed songs for the movie
\item[(also songs)]   The composer composed also songs for the movie
\end{Ventry}



\subsubsection{Key to attributes in the running time field}
\begin{Ventry}{(Video Premiere)}
\item[(VV)]        Video Version
\item[(TVV)]      TV Version
\item[(LD)]        Laser Disc Version
\item[(CD-i)]      CD-i Version
\item[(O)]         Other media, can add explicit code if enough entries

\item[(Video Premiere)]         Movie never released theatrically
\item[(Director's Cut)]         Director's Cut
\item[(Special Edition)]        Special Edition
\item[(Re-release)]            Rereleased, probably after a ban or cut
\end{Ventry}

\subsubsection{Key to certificates}

This field is used to record the Certificate awarded the movie. 
Different countries have different Certificates.
The table below lists the valid entries.

\vspace{4mm}

Australia:
\nopagebreak

\vspace{2mm}       
\begin{Ventry}{(Video Premiere)}
\item[G]      General
\item[PG]    Parental Guidance for under 15
\item[M]      15+ Recommended
\item[MA]     15+ Restricted (Under 15 only with an adult)
\item[R]      Passed only for persons 18 and over
\item[X]      Sexually Explicit, 18s and over (Possibly only videos) 
\end{Ventry}

\vspace{4mm}

Finland:
\nopagebreak

\vspace{2mm}
\begin{Ventry}{(Video Premiere)}
\item[S]     Suitable for all
\item[K--8]     Passed only for persons 8 and over
\item[K--8/K--5]     Passed only for persons 8 and over
                 (of 5 and over with an adult)
\item[K--10/K--7]    Passed only for persons 10 and over
                 (or 7 and over with an adult)
\item[K--10]     Passed only for persons 10 and over
\item[K--12]     Passed only for persons 12 and over
\item[K--12/K--9]     Passed only for persons 12 and over
                 (or 9 and over with an adult)
\item[K--13]     Passed only for persons 13 and over
\item[K--14]     Passed only for persons 14 and over
\item[K--16]     Passed only for persons 16 and over
\item[K--18]     Passed only for persons 18 and over
\end{Ventry}

\vspace{4mm}

France:
\nopagebreak

\vspace{2mm}
\begin{Ventry}{(Video Premiere)}
\item[--12]     Passed only for persons 12 and over
\item[--17]     Passed only for persons 17 and over
\item[X]      For pornographic movies
\end{Ventry}

\vspace{4mm}

Germany:
\nopagebreak

\vspace{2mm}
\begin{Ventry}{(Video Premiere)}
\item[6]      Passed only for persons 7 and over
\item[12]     Passed only for persons 12 and over
\item[16]     Passed only for persons 16 and over
\item[18]    Passed only for persons 18 and over
\item[U]      Unrated
\end{Ventry}

\vspace{4mm}

Norway:
\nopagebreak

\vspace{2mm}
\begin{Ventry}{(Video Premiere)}
\item[7]      Passed only for persons 7 and over (4 with adult)
\item[11]     Passed only for persons 11 and over (7 with adult)
\item[15]     Passed only for persons 15 and over (12 with adult)
\item[18]     Passed only for persons 18 and over
\end{Ventry}
            
\vspace{4mm}

New Zealand:
\nopagebreak

\vspace{2mm}
\begin{Ventry}{(Video Premiere)}
\item[G]     Universal. suitable for all
\item[GY]     More suitable for 13+
\item[GA]     More suitable for adults
\item[RP13]     Entrance below 13 only with parent or guardian
\item[RP16]     Entrance below 16 only with parent or guardian
\item[RP18]     Entrance below 18 only with parent or guardian
\item[R13]     Passed only for persons 13 and over
\item[R16]     Passed only for persons 16 and over
\item[R18]     Passed only for persons 18 and over
\end{Ventry}

\vspace{4mm}            

UK:
\nopagebreak

\vspace{2mm}
\begin{Ventry}{(Video Premiere)}
\item[E]      Exempt from classification
\item[U]      Universal, suitable for all
\item[PG]     Parental Guidance, some scenes may be unsuitable for kids
\item[12]     Passed only for persons 12 and over
\item[15]     Passed only for persons 15 and over
\item[18]     Passed only for persons 18 and over
\end{Ventry}

\vspace{4mm} 

US:   
\nopagebreak

\vspace{2mm}
\begin{Ventry}{(Video Premiere)}
\item[G]      For all audiences
\item[PG]     Parental Guidance Suggested (mainly for under 10's)
\item[PG--13]     Parental Guidance Suggested for children under 13
\item[R]      Under 17 not admitted without parent or guardian
\item[NC--17]     Under 17 not admitted
\end{Ventry}

\vspace{4mm}
As well as the country specific certificates, there is also a special
global certificate - Banned. This denotes any movie that has
been banned from general release.

\vspace{4mm}
Valid attributes are:

\vspace{2mm}
\begin{Ventry}{(Video Premiere)}
         \item[(VV)]         Video Version
         \item[(VP)]         Video Premiere -- Movie had no prior theatrical release
         \item[(TVV)]       TV Version
         \item[(LD)]         Laser Disc Version
         \item[(CD-i)]       CD-i Version
         \item[(O)]          Other media, can add explicit code if enough entries

         \item[(DC)]         Director's Cut
         \item[(SE)]         Special Edition
         \item[(RR)]         Rereleased, probably after a ban or cut
         \item[(CUT)]        Movie Cut/Dubbed for release on this media
\end{Ventry}      
   
\subsubsection{Key to movie ratings}

The movie ratings report is a massive on-going poll of movies rated on a scale
from 1 (bad) to 10 (excellent) by users on the internet. An example entry on
the report is:
\begin{quote}
{\tt 0000000123     506   8.5  Casablanca}
\end{quote}
     
For each movie, the vote distribution, the total number of votes and the 
average rating are shown. The vote distribution uses a single character to 
represent the percentage of votes for each ranking.  The following characters
codes can appear:

\begin{tabular}{ll}
. & no votes cast\\        
0 & 1--9\%  of the votes\\
1 & 10--19\% of the votes\\
2 & 20--29\% of the votes\\
3 & 30--39\% of the votes\\
4 & 40--49\% of the votes\\
5 & 50--59\% of the votes\\
6 & 60--69\% of the votes\\
7 & 70--79\% of the votes\\
8 & 80--89\% of the votes\\
9 & 90--99\% of the votes\\
$\ast$ & 100\%   of the votes\\
\end{tabular}

\vspace{4mm}
So, in our example above, "Casablanca" has received a total of 506 votes, the
average of the votes being 8.5 with 30--39\% 10/10; 20--29\% 9/10; 10--19\% 8/10; 
1--9\% 7/10 etc.

       
\end{document}

